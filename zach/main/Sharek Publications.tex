% resume.tex
% Zachariah Sharek
% Based on a resume found at www.toofishes.net/blog/latex-resume-follow-up/

\documentclass[12pt,letterpaper,oneside]{article}
\usepackage[letterpaper,left=0.75in,right=0.85in,top=0.75in,bottom=0.5in]{geometry}
\usepackage[utf8]{inputenc}
\usepackage{mdwlist}
\usepackage[T1]{fontenc}
\usepackage{textcomp}
\usepackage{mathpazo}
\usepackage{tgpagella}
\usepackage{enumitem}
\usepackage{xspace}
\setlist[itemize]{leftmargin=6mm}
\setdescription{leftmargin=0pt}

% These next lines will set the resume in a nice sans-serif font (but without non-lining numbers, unfortunately)
%\renewcommand{\sfdefault}{Myriad-LF}
%\renewcommand{\familydefault}{\sfdefault}

% to make ligatures searchable
% the file glyphtounicode.tex can be in the same directory 
% as the main tex file or in the local latex file directory
\input{glyphtounicode}
\pdfgentounicode=1

\pagestyle{empty}
\setlength{\tabcolsep}{0em}
%\hyphenpenalty=10000

% indentsection style, used for sections that aren't already in lists
% that need indentation to the level of all text in the document
\newenvironment{indentsection}[1]%
{\begin{list}{}%
	{\setlength{\leftmargin}{#1}}%
	\item[]%
}
{\end{list}}

% opposite of above; bump a section back toward the left margin
\newenvironment{unindentsection}[1]%
{\begin{list}{}%
	{\setlength{\leftmargin}{-0.5#1}}%
	\item[]%
}
{\end{list}}

% format two pieces of text, one left aligned and one right aligned
\newcommand{\headerrow}[2]
{\begin{tabular*}{\linewidth}{l@{\extracolsep{\fill}}r}
	#1 &
	#2 \\
\end{tabular*}}

% and the actual content starts here
\begin{document}

\begin{center}
{\LARGE \textbf{\textsc{Zachariah S. Sharek}}}
\vspace{4mm}

%\headerrow
%	{134 S Fairmount St\ \ \textbullet \ \ Pittsburgh, PA 15206}
%	{sharek@gmail.com\ \ \textbullet \ \ (919) 757\textendash 4569}	
\end{center}
\vspace{-2mm}

\hrule
\vspace{-0.4em}
\subsection*{\centering\textbf{Publications}}

\begin{itemize}
	\parskip=0.1em
	\item Swift, S. A., Moore, D. A., Sharek, Z. S., \& Gino, F. (2013). 
	Inflated applicants: Attribution errors in performance evaluation by professionals. 
	\emph{PLoS One}, 8(7), e69258.
\end{itemize}

\begin{itemize}
	\parskip=0.1em
	\item Sharek, Z. S., Schoen, R. E. \& Loewenstein, G. (2012). Bias in the Evaluation of Conflict of Interest Policies.  
	\emph{Journal of Law, Medicine \& Ethics}, 40(2), 368-382.
\end{itemize}

\begin{itemize}
	\parskip=0.1em
	\item Gino, F., Sharek, Z. S., Moore, D. A. (2011).  Keeping the illusion of control under control: Ceilings, floors, and imperfect calibration.  \emph{Organizational Behavior and Human Decision Processes}, 114(2), 104-114.   
\end{itemize}

\begin{itemize}
	\parskip=0.1em
	\item Moore, D. A., Swift, S. A., Sharek, Z. S., \& Gino, F. (2010). Correspondence bias in performance evaluation: Why grade inflation works. \emph{Personality and Social Psychology Bulletin}, 36(6), 843-852.
\end{itemize}

\hrule
\vspace{-0.4em}
\subsection*{\centering\textbf{Publications under Review or Preparation}}

\begin{itemize}
	\parskip=0.1em
	\item Swift, S. A., Sharek, Z. S., Gino, F., \& Moore, D. A. On the Robustness and Generality of the
	Correspondence Bias. Under review at \emph{Journal of Consumer Research}.	
\end{itemize}

\begin{itemize}
	\parskip=0.1em
	\item Sharek, Z. S. \& Moore, D. A. The Illusion of the Illusion of Control. 
	In preparation for submission to \emph{Psychological Review}. This is adapted from my dissertation.	
\end{itemize}

\hrule
\vspace{-0.4em}
\subsection*{\centering\textbf{Conference Presentations \& Proceedings}}

\begin{itemize}
\parskip=0.1em
\item Sharek, Z. S., Swift, S. A., Moore, D. A., \& Gino, F. (2010, June) On the robustness and generality of the correspondence bias. Behavioral Decision Research in Management, Pittsburgh, PA. 
\item Swift, S. A., Sharek, Z. S., Moore, D. A., \& Gino, F. (2010, June) Seeing through performance: Attribution errors in performance evaluation by experts. Behavioral Decision Research in Management, Pittsburgh, PA.
\item Sharek, Z. S., Swift, S. A., Gino, F., \& Moore, D. A. (2009, October). On the robustness and generality of the correspondence bias. Association for Consumer Research, Pittsburgh, PA.
\item Moore, D. A., Swift, S. A., Sharek, Z. S., \& Gino, F. (2008, November). Correspondence bias in 
performance evaluation and the benefits of having been graded leniently. Society of Judgment and Decision Making Conference, Chicago, Illinois.   
\item Gino, F., Sharek, Z. S., \& Moore, D. A. (2008, October).  The illusion of the illusion of control.  
Association for Consumer Research, San Francisco, California.   
\item Sharek, Z. S., Moore, D. A., Swift, S. A., \& Gino, F. (2008, August). Reducing, enhancing and eliminating the correspondence bias. Academy of Management, Anaheim, California.
\item Sharek, Z. S., Moore, D. A., Swift, S. A., \& Gino, F., (2007, August).  Correspondence bias in 
performance evaluation.  Academy of Management, Philadelphia, Pennsylvania.  
\end{itemize}
\end{document}
